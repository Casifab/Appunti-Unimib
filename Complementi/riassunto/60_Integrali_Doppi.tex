\chapter{Integrali doppi}

\section{Generalità}

$$\iint_A f(x,y) \dif x \dif y$$

$A$ è l'insieme (zona) di integrazione ed è un \textbf{sottoinsieme limitato} del piano.

$f(x,y) : A \to R$ è una funzione limitata. Esiste cioè un $M$ tale che $|f(x,y)| \leq M$ $\forall (x,y) \in A$.

\section{Significato geometrico}

\subsection{In $\R$}

$$\int_a^b f(x) \dif x$$

L'integrale è interpretabile geometricamente come l'area con segno della parte di piano sottesa dalla funzione $f(x)$.

\subsection{In $\R^2$}

$$\iint_A f(x,y) \dif x \dif y$$

Se $f(x,y) \ge 0$ allora l'integrale è il volume della parte di spazio compresa tra il piano $xy$ ed il grafico di $f(x,y)$, limitata dalle \textit{pareti verticali} che si proiettano sul bordo di $A$.

Altrimenti, se $f(x,y)$ ha segno variabile in $A$, ``le parti'' al di sotto del piano $xy$ contano con il segno $-$.

\section{Definizione}

\subsection{Funzione costante su un rettangolo}

$A= [a,b] \times [c,d]$

Cioè $A$ è un rettangolo delimitato dai lati $ab$ e $cd$.

$f(x,y) = \lambda$, cioè una costante.

Quindi l'integrale è semplicemente $$\lambda \cdot (b-a) \cdot (d-c)$$ ovvero il volume con segno del parallelepipedo.

\subsection{Funzioni costanti su più rettangoli}

Sia $A$ un'unione disgiunta di rettangoli (intuitivamente come tanti grattacieli uno vicino all'altro).

$f(x,y)$ è costante all'interno di ciascun rettangolo, ma può variare da rettangolo a rettangolo.

$f(x,y) = \lambda_i$ $\forall (x,y) \in Rett_i$.

Quindi l'integrale é $$\sum_{i=1}^{n} \lambda_i \cdot \text{Area}(Rett_i)$$

\subsection{Funzione non costante su un rettangolo}

Sia $D$ un rettangolo chiuso, con lati paralleli agli assi, $f(x,y)$ è una funzione qualsiasi definita sul dominio $D$.

Supponiamo che $D$ vari tra $a$ e $b$ sull'asse $x$ e tra $c$ e $d$ sull'asse $y$.

Possiamo quindi partizionare il dominio in dei rettangoli più piccoli, individuando dei punti in questo modo:

$$a = x_0 < x_1 < \ldots < x_{m-1} < x_m = b$$


$$c = y_0 < y_1 < \ldots < y_{n-1} < y_n = d$$

In sostanza si individuano $m \cdot n$ rettangoli $Rett_{ij}$.

L'area di ogni rettangolo è $$\delta A_{ij} = \delta x_i \delta y_1$$

La lunghezza della diagonale è
$$\sqrt{(\delta x_i)^2+(\delta y_i)^2}$$

Scelto un punto arbitrario $(x_{ij},y_{ij})$ in ciascuno dei rettangoli individuiamo la somma di Riemann:
$$R(f,P_{artizione}) = \sum^m_{i=1} \sum^n_{j=1} f(x_{ij},y_{ij}) \delta A_{ij}$$

In sostanza base per altezza di un parallelepipedo.

Facendo il limite
$$\lim_{(n,m) \to (\infty, \infty)} R(f,P_{artizione})$$ otteniamo l'integrale doppio di $f$ su $D$, cioè il volume sopra $D$ e sotto il grafico di $f$.


\section{Funzioni integrabile}

\begin{definition}
Si dice che $f$ è integrabile sul rettangolo $D$ e che ha integrale doppio $$I = \iint_D f(x,y) \dif A$$
se, per ogni numero positivo $\epsilon$, esiste un numero $\gamma(\epsilon)$ tale  che $$R(f,P)-I<\epsilon$$ vale per ogni $P$ di $D$ tale che $|P|<\gamma$ e per tutte le scelte di punti $(x_{ij},y_{ij})$.
\end{definition}

\begin{property}
Se una funzione è continua su $D$ allora è integrabile su $D$. Ovviamente, non è detto che se $f$ è integrabile allora è continua.
\end{property}

\section{Integrali doppi su domini generali}

Potrebbe anche essere necessario calcolare un integrale su un dominio $D$ che non è un rettangolo ma un'area qualunque del piano.
