\chapter{Derivate}

A partire dal problema dell'individuazione della tangente geometrica ad una curva sono state formulate le basi del calcolo differenziale.

\section{Rapporto incrementale}

Sia $f: (a,b) \to \mathbb{R}$ e sia $x_0 \in (a,b)$. Sia $x_0 + h$, con $h \neq 0$, un altro punto di $(a,b)$. Si chiama rapporto incrementale della funzione $f$, con punto iniziale $x_0$ e incremento $h$ della variabile dipendende, la quantità $$\frac{f(x_0+h)-f(0)}{h}$$

\subsection{Significato geometrico}
Il rapporto incrementale costituisce il coefficiente angolare della retta (secante) passante per i punti $(x_0,f(x_0))$ e $(x_0+h,f(x_0+h))$. Infatti, per ogni $h$ scelto la secante ha equazione $$f(x_0) + \frac{f(x_0+h)-f(x_0)}{h}\cdot (x-x_0)$$

Se la funzione è quindi derivabile, se $h \to 0$ il punto $(x_0+h,f(x_0+h))$ tende al punto $(x_0,f(x_0))$ e quindi la retta secante diventa retta tangente di $f$ nel punto $(x_0,f(x_0))$

\section{Derivata}
Sia $f: (a,b) \to \mathbb{R}$ e sia $x_0 \in (a,b)$. Si dice che la funzione è derivabile in $x_0$ se esiste, finito,  il limite del rapporto incrementale con $h \to 0$. Tale limite si indica con $f'(x_0)$ e si chiama derivata di $f$ in $x_0$.

$$f'(x_0)=\lim_{h\to 0} \frac{f(x_0+h)-f(x_0)}{h}$$

Se si pone $x=x_0+h$, il rapporto incrementale diventa 
$$f'(x)=\lim_{x\to x_0} \frac{f(x)-f(x_0)}{x-x_0}$$


\subsection{Retta tangente}

Sia $f:(a,b)\to\mathbb{R}$ derivabile in $x_0 \in (a,b)$. Si chiama retta tangente al grafico della funzione in $(x_0,f(x_0))$ la retta $$y = f(x_0) + f'(x_0)(x-x_0)$$

\section{Relazione tra derivabilità e continuità}

\textbf{Enunciato}

Se una funzione è derivabile in un punto $x_0$ allora è anche continua in quel punto.

Cioè $lim_{x\to x_0}$
\begin{tip}
\textit{Derivabile in $x_0$}

$\lim_{x\to x_0} \frac{f(x)-f(x_0)}{x-x_0} = f'(x_0)$

\textit{Continua in $x_0$}

$\lim_{x\to x_0} f(x) = f(x_0)$

\end{tip} 

\textbf{Dimostrazione}

Sappiamo che $$f(x) = f(x_0) + \frac{f(x)-f(x_0)}{x-x_0} (x-x_0)$$

Calcoliamo il limite per $x \to x_0$ di entrambi i membri.

$\lim_{x\to x_0} f(x) = \lim_{x\to x_0} f(x_0) + \lim_{x\to x_0} \frac{f(x)-f(x_0)}{x-x_0} \cdot \lim_{x\to x_0} (x-x_0)$

Quindi

$\lim_{x\to x_0} f(x_0) = f(x_0)$

$\lim_{x\to x_0} \frac{f(x)-f(x_0)}{x-x_0} = f''(x_0)$ perchè la funzione è derivabile per ipotesi in $x_0$.

$lim_{x\to x_0} (x-x_0) = x_0 - x_0 = 0$

Quindi 

$\lim_{x\to x_0} f(x) = f(x_0)$

\section{Regole di derivazione}
Se $f,g: (a,b) \to \mathbb{R}$ sono due funzioni derivabili in $(a,b)$, allora:

\begin{itemize}
\item $f+g$ è derivabile in $(a,b)$ e $(f+g)' = f' + g'$
\item dato $a \in \mathbb{R}$. $a \cdot f$ è derivabile e $(a\cdot f)' = a \cdot f'$
\item $f*g$ è derivabile in $(a,b)$ e $(f*g)' = f' * g'$
\item $g(x) \neq 0$ per ogni $x \in (a,b)$ allora $\frac{f}{g}$ è derivabile e $(\frac{f}{g})' = \frac{f'g - fg'}{g^2}$ 
\end{itemize}

\section{Derivate funzioni elementari}

\subsection{Derivata di $\sin(x)$}
Sia $f(x) = \sin(x)$, allora $f'(x) = \cos(x)$.

\textbf{Dimostrazione}

$$\lim_{h\to 0} \frac{\sin(x+h)-\sin(x)}{h} \\
=\lim_{h\to 0} \frac{\sin(x)\cos(h)+\cos(x)\sin(h)-\sin(x)}{h} \\
=\lim_{h\to 0} (\sin(x)\frac{\cos(h)-1}{h} + \cos(x)*\frac{\sin(h)}{h}$$

Sappiamo che $1 - \cos(h) \sim \frac{1}{2} h^2$, quindi $\lim \frac{1-\cos(h)}{h} = 0$.

$$= \cos(x) \cdot \lim_{h \to 0} \frac{\sin(h)}{h} = \cos(x)$$

\subsection{Derivata di $\cos(x)$}
Sia $f(x) = \cos(x)$, allora $f'(x) = -\sin(x)$.

\textbf{Dimostrazione}

$$\lim_{h \to 0} \frac{\cos (x+h) - \cos x}{h} $$ $$
=  \lim_{h \to 0} \frac{\cos x \cos h - \sin x \sin h - \cos x}{h} $$ $$
=  \cos x \cdot \underbrace{\lim_{h \to 0} \frac{\cos h - 1}{h}}_{0} - \sin x \cdot \underbrace{\lim_{h \to 0} \frac{\sin h}{h}}_{1} $$ $$
=  - \sin x$$

\subsection{Derivata di $\tan(x)$}

Sia $f(x) = \tan(x)$, allora $f'(x) = \frac{1}{\cos^2(x)}$.

\textbf{Dimostrazione}

Conoscendo le derivate di $\sin(x)$ e $\cos(x)$, scriviamo $f(x) = \tan(x) = \frac{\sin(x)}{\cos(x)}$. Applichiamo la regola di derivazione del rapporto:

$$f'(x) = \frac{\cos x \cdot \cos x - \sin x \cdot (-\sin x)}{\cos^2 x} = \frac{1}{\cos^2x}$$

\subsection{Derivata di $e^x$}
Se $f(x) = e^x$, allora $f'(x) = e^x$.

\textbf{Dimostrazione}
$$
\lim_{h \to 0} \frac{e^{x+h} - e^x}{h} $$ $$
= \lim_{h \to 0} \frac{e^x \cdot (e^h - 1)}{h} $$ $$
= e^x \cdot \underbrace{\lim_{h \to 0} \frac{e^h - 1}{h}}_{1} = e^x
$$

\subsection{Derivata di $\log(x)$}
Se $f(x) = \log x$, allora $f'(x) = \frac{1}{x}$. Infatti:

\textbf{Dimostrazione}
$$\lim_{h \to 0} \frac{\log (x+h) - \log x}{h} $$ $$
= \lim_{h \to 0} \frac{\log \frac{x+h}{x}}{h} $$ $$
= \lim_{h \to 0} \frac{1}{h} \cdot \log \left(1 + \frac{h}{x} \right) $$

Osservando che $\frac{h}{x}$ tende comunque a zero, possiamo applicare il limite notevole $\log (1 + \frac{h}{x} ) \sim \frac{h}{x}$.

$$= \lim_{h \to 0} \frac{1}{h} \cdot \frac{h}{x} = \frac{1}{x}$$

\section{Punti di non derivabilità}

Ricordiamo che una funzione $y=f(x)\mbox{, }f:\mathbb{R}\to \mathbb{R}$ è derivabile in un punto $x_0$ se esistono finiti e uguali i limiti sinitro e destro del rapporto incrementale. $$\lim_{h\to 0^{+}}{\frac{f(x_0+h)-f(x_0)}{h}}=\lim_{h\to 0^{-}}{\frac{f(x_0+h)-f(x_0)}{h}} \lim_{h\to 0^{-}}{\frac{f(x_0+h)-f(x_0)}{h}}=\lim_{h\to 0^{-}}{\frac{f(x_0+h)-f(x_0)}{h}}$$

Adattare la definizione di punti di discontinuità per le derivate.

\section{Teorema di Fermat}

\textbf{Enunciato}
Sia $f: (a,b)\to\textbf{R}$ una funzione e si supponga che $x_0 \in (a,b)$ sia un punto di estremo locale (massimo o minimo) di $f$. Se $f$ è derivabile nel punto $x_0$, allora $f'(x_0)=0$.

\textbf{Dimostrazione}

Supponiamo che $x_0$ sia un punto di minimo relativo. Esiste $\delta > 0$ tale che per ogni $x \in (x_0-\delta, x_0+\delta)$ si abbia $f(x)\geq f(x_0)$

Per $x_0<x<x_0+\delta$ (rapporto incrementale destro)  si ha $$ \frac{f(x)-f(x_0)}{x-x_0} \geq 0$$

Per $x_0-\delta<x<x_0$ (rapporto incrementale sinistro) si ha $$ \frac{f(x)-f(x_0)}{x-x_0} \leq 0$$

Visto che la funzione $f(x)$ è derivabile in $x_0$ (per ipotesi), calcoliamo il limite destro del rapporto incrementale $$f'(x_0) = \lim_{x\to x^+_0}\frac{f(x)-f(x_0)}{x-x_0} \geq 0$$

e anche il limite sinistro del rapporto incrementale $$f'(x_0) = \lim_{x\to x^-_0}\frac{f(x)-f(x_0)}{x-x_0} \leq 0$$

Combinando le due soluzioni ($f'(x_0)\leq 0 \land f'(x_0)\geq 0$) per forza deve essere che $f'(x_0)=0$


\section{Teorema di Rolle}

\textbf{Enunciato}

Presa $f: [a,b] \to \mathbb{R}$, se essa è continua in $[a,b]$, derivanbile in $(a,b)$ e $f(a)=f(b)$, allora esiste un punto $z \in (a,b)$ tale che $f'(z)=0$.

\textbf{Dimostrazione}

La funzione è continua in un intervallo, per il teorema di Weierstrass esiste per forza un punto di massimo assoluto e minimo assoluto in $[a,b]$.   

Se il punto trovato è sia di massimo che di minimo allora la funzione è costante.

Altrimenti:

\begin{itemize}
\item Se $a$ non è un punto di massimo allora neanche $b$ lo è, visto che $f(a)=f(b)$. Quindi un punto di massimo ci deve essere per forza in $(a,b)$ in cui, per il teorema di Fermat, la derivata vale 0.

\item Se $a$ non è un punto di minimo allora neanche $b$ lo è, visto che $f(a)=f(b)$. Quindi un punto di minimo ci deve essere per forza in $(a,b)$ in cui, per il teorema di Fermat, la derivata vale 0.
\end{itemize}


\section{Teorema di Lagrange}

\textbf{Enunciato}
Presa $f: [a,b] \to \mathbb{R}$, se essa è continua in $[a,b]$, e derivabile in $(a,b)$. Allora esiste un $x \in (a,b)$ tale che $$f'(x)=\frac{f(b)-f(a)}{b-a}$$

\textbf{Dimostrazione}

Consideriamo la funzione
\begin{equation*}
g(x) = f(x) - (x-a) \cdot \frac{f(b)-f(a)}{b-a}
\end{equation*}
Questa funzione soddisfa le ipotesi del teorema di Rolle, infatti:
\begin{equation*}
g(b) = f(b) - \cancel{(b-a)} \cdot \frac{f(b)-f(a)}{\cancel{b-a}} = f(a)
\end{equation*} 
e $g(a) = f(a)$ per gli stessi calcoli. Inoltre $g$ è continua in $[a,b]$ e derivabile in $(a,b)$ perché $f$ lo è.

Il teorema di Rolle ci garantisce l'esistenza di almeno un $x \in (a,b)$ tale che $g'(x)=0$. Calcoliamo la derivata della funzione:
\begin{equation*}
g'(x) = f'(x) - \frac{f(b)-f(a)}{b-a}
\end{equation*}
Quindi
\begin{equation*}
f'(x) = \underbrace{g'(x)}_{0} + \frac{f(b)-f(a)}{b-a}
\end{equation*}
\begin{equation*}
f'(x) = \frac{f(b)-f(a)}{b-a}
\end{equation*}

\section{Teorema di Cauchy}
Siano $f, g : [a,b] \to \mathbb{R}$ due funzioni continue in $[a,b]$ e derivabili in $(a,b)$. 

Allora esiste un punto $c\in (a,b)$ tale che

$$[g(b)-g(a)]f'(c) = [f(b) - f(a)]g'(c).$$

\section{Teorema de L'Hopital}

Siano $f, g : [a,b] \rightarrow \mathbb{R}$ due funzioni continue in $[a,b]$ e derivabili in $(a,b)$, con $-\infty \leq a < b \leq +\infty$; sia $g^{\prime}(x)$ diversa da 0 in ogni punto di tale intervallo, tranne al più in $c \in (a,b)$. Sia inoltre

$$\lim_{x\to c}{f(x)} = \lim_{x\to c}g(x) = 0$$

oppure 

$$\lim_{x\to c}{|f(x)|} = \lim_{x\to c}{|g(x)|} = \infty,$$

ed esista 

$$ L = \lim_{x \to c}{\frac{f'(x)}{g'(x)}} \in \mathbb{\bar{R}} $$

Allora

$$\lim_{x \to c}{\frac{f(x)}{g(x)}} = L$$

\section{Polinomio di Taylor}

Data una funzione $f: I\to \mathbb{R}$, derivabile $n-1$ volte, con $f^{(n-1)\prime}$ derivabile in $x_0$, si definisce polinomio di Taylor $$P_n(x)=\serie{k=0}{n}\frac{1}{k!}\cdot f^{k\prime}(x_0)(x-x_0)^k$$


\subsection{Resto in forma di Peano}

Data una funzione $f: I\to \mathbb{R}$, derivabile $n-1$ volte, con $f^{(n-1)\prime}$ derivabile in $x_0$, allora $$f(x)=P_n(x)+o((x-x_0)^n)$$

La funzione si può quindi esprimere tramite un polinomio di Taylor e un resto $R_n(n)=f(x)-P_n(x)$, tale che $$\lim_{x\to x_0} \frac{R_n(x)}{(x-x_0)^n}=0$$

\subsection{Resto in forma di Lagrange}

Data una funzione $f: I\to \mathbb{R}$, derivabile $n-1$ volte, definita su I intervallo aperto. Per ogni $x,x_0 \in I$ esiste $c$ compreso tra $x$ e $x_0$ tale che $$f(x)-P_n(x)=\frac{f^{(n+1)\prime}(c)}{(n+1)!}\cdot (x-x_0)^{n+1} = \text{Resto di Lagrange}$$ 

\section{Convessità e concavità}

\subsection{Funzione Concava}
Dato un intervallo $I$, $f: I \to \mathbb{R}$ si dice \emph{convessa} se $\forall x_1, x_2 \in I$ vale
\begin{equation*}
f(x_1 + t(x_2-x_1)) \le f(x_1) + t(f(x_2)-f(x_1))
\end{equation*}
per ogni $t \in [0,1]$.

\begin{center}
\begin{tikzpicture}[scale=1.5]
\draw[->] (-2.2,0) -- (2.2,0) node[right] {$x$};
\draw[->] (0,-0.5) -- (0,2.2) node[above] {$y$};
\draw [smooth,domain=-2:2] plot({\x}, {0.25*\x*\x});
\draw [smooth,domain=0:1.5] plot({\x}, {0.375*\x});
\end{tikzpicture}
\end{center}

Di fatto questo significa che il grafico della funzione sta sotto la corda.

\subsection{Funzione Convessa}

Con definizione analoga, si dice che $f$ è \emph{strettamente convessa} se vale
\begin{equation*}
f(x_1 + t(x_2-x_1)) < f(x_1) + t(f(x_2)-f(x_1))
\end{equation*}

Invertendo le due disuguaglianze precedenti si ottengono intuitivamente le definizioni di funzione \emph{concava} e \emph{strettamente concava}:
\begin{equation*}
f(x_1 + t(x_2-x_1)) \ge f(x_1) + t(f(x_2)-f(x_1))
\end{equation*}

\begin{center}
\begin{tikzpicture}[scale=1.5]
\draw[->] (-2.2,0) -- (2.2,0) node[right] {$x$};
\draw[->] (0,-0.5) -- (0,2.2) node[above] {$y$};
\draw [smooth,domain=-1.75:1.75] plot({\x}, {-0.25*\x*\x+1});
\draw [smooth,domain=0:1.5] plot({\x}, {-0.375*\x+1});
\end{tikzpicture}
\end{center}

Diamo ora per vero che una funzione derivabile è convessa se e solo se la sua derivata è non decrescente.

Se $f'$ è non decrescente, allora $g(x) = f(x)-f'(x_0)(x-x_0) + f(x_0)$ ha un minimo in $x_0$.

Infatti $g'(x) = f'(x)-f'(x_0)$; quindi $g'(x) \ge 0$ per $x > x_0$ e $g'(x) \le 0$ per $x < x_0$. Quindi effettivamente esiste un minimo in $x_0$.

\begin{center}
\begin{tikzpicture}[scale=1.5]
\draw[->] (-1.2,0) -- (2.2,0) node[right] {$x$};
\draw[->] (0,-0.5) -- (0,2.2) node[above] {$y$};
\draw [smooth,domain=-1:2] plot({\x}, {0.25*\x*\x+0.5});
\draw [smooth,domain=0.1:2] plot({\x}, {0.5*\x+0.25});
\end{tikzpicture}
\end{center}

La retta tangente ha equazione $y = f(x_0)+f'(x_0)(x-x_0)$. Consideriamo $g(x)$ come differenza tra la funzione e la retta tangente. A questo punto $g(x_0) = f(x_0) - f(x_0) = 0$, quindi $g(x) \ge 0$ $\forall x$.


Sia $f: I \to \mathbb{R}$ derivabile, allora $f$ è convessa se e solo se $f'$ è non decrescente.

Se $f: I \to \mathbb{R}$ è derivabile due volte, allora è convessa se e solo se $f''(x) \ge 0$ in ogni punto.

\section{Punti di flesso}

Si dice che $x$ è un \emph{punto di flesso} per $f$ se $f$ è concava in $(x, x + \delta)$ e $f$ è convessa in $(x-\delta, x)$ o viceversa.

Nei punti di flesso la tangente attraversa il grafico della funzione.

In un punto di flesso la funzione cambia concavità.

\subsection{Punti a tangente verticale}

Si dice che $x_0$ è un \emph{punto a tangente verticale} per $f$ se
\begin{equation*}
\lim_{x \to x_0} \frac{f(x)-f(x_0)}{x-x_0}
\end{equation*}
è infinito.
