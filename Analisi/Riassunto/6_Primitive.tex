\newcommand*\dif{\mathop{}\!\mathrm{d}}

\chapter{Primitive}

\section{Definizioni}

\subsection{Primitiva}

Data una funzione $f: (a,b) \to \mathbb{R}$, si dice che $F$ è una primitiva di $f$ se $F'=f$

\subsection{Integrale indefinito}

Si dice integrale indefinito di $f(x)$ l'insieme di tutte le primitive di $f(x)$ e si scrive come $\int f(x) \dif x$.

\begin{tip}
$F(x)$ è una primitiva di $f(x)$ se e solo e $F(x) \in \int f(x) \dif x$
\end{tip}

\section{Linearità dell'integrale}

Se $f,g$ ammettono una primitiva, allora anche la somma ammette una primitiva.

$$\int (f(x)+g(x)) \dif x = \int f(x)\dif x + \int g(x)\dif x$$

Se $\lambda \in \mathbb{R}$ e $f(x)$ ha una primitiva $F(x)$, allora

$$\int \lambda f(x) \dif x = \lambda \int f(x)\dif x$$

\section{Integrazione per parti}
Siano $f, g: I \to \mathbb{R}$ due funzioni. Sia $f$ derivabile; supponiamo che $g$ abbia una primitiva $G$ e $f \cdot g$ abbia una primitiva. Allora

$$\int f(x)g(x) \dif x = f(x)G(x) - \int f'(x) G(x) \dif x$$

\section{Sostituzione di variabile}

Sia $f: I \to \mathbb{R}$ e $g: J \to \mathbb{R}$ derivabile; con $I$ e $J$ intervalli tali che abbia senso la scrittura $f(g(x))$. Se $f(x)$ ha una primitiva $F(x)$ allora

$$F(g(t)) = \int f(g(t)) \cdot g'(t) \dif x$$

\begin{tip}
In altri termini, posto $x = g(t)$:
$\int f(x) \dif x = \int f(g(t)) \cdot g'(t) \dif t \\
= \int f(x) \cdot g'(t) \dif t$
\end{tip}


\section{Integrali di funzioni fratte}

(TODO?)

\chapter{Integrali definiti}

\section{Partizioni}

Sia $[a,b]$ un intervallo chiuso e limitato. Una partizione $P$ di $[a,b]$ è un insieme di $n+1$ punti $$a=x_0<x_1<\ldots <x_n=b$$

\begin{tip}
Una partizione P suddivide l'intervallo $[a,b]$ in $n$ intervalli.
\end{tip}
    
\subsection{Somma inferiore}

Se $f$ è una funzione $[a,b] \to \mathbb{R}$ limitata e $P$ è una partizione di $[a,b]$, la somma inferiore di $f$ relativa a $P = \{x_0, \ldots, x_n\}$ è $$ s(f,P) = \sum_{i=1}^n \underbrace{(x_i-x_{i-1})}_{base} \cdot \underbrace{m_i}_{altezza} $$ dove $$ m_i = \inf_{x_{i-1} \le x \le x_i} f(x) $$

\begin{tip}
Prendo ogni singola base del rettangolo che nasce dal partizionamento e lo moltiplico per il punto minimo di $f(x)$ compreso tra gli estremi di ogni singola base.

\textbf{Aggiungi foto}
\end{tip}

\subsection{Somma superiore}
Se $f$ è una funzione $[a,b] \to \mathbb{R}$ limitata e $P$ è una partizione di $[a,b]$, la somma superiore di $f$ relativa a $P = \{x_0, \ldots, x_n\}$ è $$ S(f,P) = \sum_{i=1}^n (x_i-x_{i-1}) \cdot M_i $$ dove $$M_i = \sup_{x_{i-1} \le x \le x_i} f(x)$$

\begin{tip}
Prendo ogni singola base del rettangolo che nasce dal partizionamento e lo moltiplico per il punto massimo di $f(x)$ tra gli estremi di ogni singola base.
\textbf{Aggiungi foto}
\end{tip}

\subsection{Raffinamento di una partizione}
Data una partizione $P$, si dice che una partizione $P^*$ è un raffinamento di $P$ se $P \subset P^*$, ossia se ogni punto di $P$ è anche punto di $P^*$. Date due partizioni $P_1$ e $P_2$ si dice comune raffinamento di $P_1$ e $P_2$ la partizione $P^* = P_1 \cup P_2$.

\begin{tip}
Un raffinamento è quindi ottenuto introducendo nuovi punti nella partizione.
\end{tip}

Sia $f:[a,b] \to \mathbb{R}$ limitata. Se $P$ è una partizione di $[a,b]$ e $P^*$ un suo raffinamento, allora $$
s(f,P) \le s(f, P^*) \qquad \text{ e } \qquad S(f,P) \ge S(f,P^*) $$

$$(b-a) \inf f \le \sup_P s(f,P) \le \inf_P S(f,P) \le (b-a) \sup f$$
\section{Integrali definiti}

\subsection{Funzione integrabile}

Una funzione $f(x)$ limitata definita su $[a,b]$ si dice integrabile se $$ \sup_P s(f,P) = \inf_P S(f,P) $$ In questo caso si dice che $$\sup_P s(f,P) = \int_a^b f(x) \dif x$$


\subsection{Integrale definito}

Si dice integrale definito tra $a,b$ di una funzione integrabile $f(x)$ $$\int_a^b f(x) \dif x$$ Tale integrale viene indicato per indicare l'area sottesa al grafico di una funzione nell'intervallo $[a,b]$. 

\subsubsection{Proprietà dell'integrale}

Sia $f:[a,b] \to \mathbb{R}$ e $c \in [a,b]$. La funzione $f$ è integrabile su $[a,b]$ se e solo se è integrabile su $[a,c]$ e  $[c,b]$; vale $$\int_a^b f(x) \, dx = \int_a^c f(x) \, dx + \int_c^b f(x) \, dx$$

Se $b \le a$ allora $$\int_a^b f(x) \, dx = - \int_b^a f(x) \, dx$$

Se $b = a$ allora $$\int_a^a f(x) \, dx = 0$$

Se $f, g:[a,b] \to \mathbb{R}$ sono funzioni integrabili, allora $f+g$ è una funzione integrabile e $$\int_a^b (f(x)+g(x)) \, dx = \int_a^b f(x) \, dx + \int_a^b g(x) \, dx$$

Se $\lambda \in \mathbb{R}$, allora $\lambda f(x)$ è una funzione integrabile e $$\int_a^b \lambda f(x) \, dx = \lambda \int_a^b f(x) \, dx$$

Se $f: [a,b] \to \mathbb{R}$ è una funzione integrabile, allora $|f(x)|$ è integrabile e $$\left\lvert \int_a^b f(x) \, dx \right \vert \le \int_a^b |f(x)| \, dx$$

\subsection{Integrabilità delle funzioni monotone}
\textbf{Enunciato}

Supponiamo che $f: [a,b] \to \mathbb{R}$ sia monotona. Allora $f$ è integrabile.

\textbf{Dimostrazione}

Supponiamo senza perdita di generalità $f$ non decrescente, allora $\forall x$ vale $f(a) \le f(x) \le f(b)$; quindi la funzione è limitata.

Sia $P_n$ la partizione $\{x_0, \ldots, x_n\}$ e chiamiamo 
\begin{equation*}
x_i - x_{i-1} = \delta = \frac{b-a}{n}
\end{equation*}
Esprimiamo la somma inferiore:
\begin{equation*}
s(f,P_n) = \sum_{i=1}^n (x_i - x_{i-1}) \cdot m_i
\end{equation*}
Ricordiamo che
\begin{equation*}
m_i = \inf_{x_{x-1}\le x \le x_i} f(x)
\end{equation*}
ma essendo $f$ non decrescente, $m_i = f(x_{i-1})$. Quindi
\begin{align*}
s(f,P_n) &= \sum_{i=1}^n (x_i - x_{i-1}) \cdot f(x_{i-1}) \\
&= \sum_{i=1}^n \delta \cdot f(x_{i-1})
\end{align*}

Analogamente si ragiona per la somma superiore e si osserva che il $\sup$ in questo caso coincide con $f(x_i)$, sempre perché $f$ è non decrescente. Quindi
\begin{equation*}
S(f,P_n) = \sum_{i=1}^n \delta \cdot f(x_i)
\end{equation*}

Calcoliamo quindi la differenza tra le due somme in questo caso particolare:
\begin{align*}
S(f,P_n) - s(f,P_n) &= \sum_{i=1}^n \delta \cdot f(x_i) - \sum_{i=1}^n \delta \cdot f(x_{i-1}) \\
&= \delta [f(x_1) + \ldots + f(x_n)] - \delta[f(x_0) + \ldots + f(x_{n-1})] \\
&= \delta [f(x_n) - f(x_0)] \\
&= \frac{b-a}{n} \cdot [f(b)-f(a)]
\end{align*}

Quindi
\begin{equation*}
\lim_{n \to +\infty} S(f,P_n) - s(f,P_n) = 0
\end{equation*}

Per definizione di limite, $\forall \epsilon > 0$ esiste $P_n$ tale che $S(f,P_n) - s(f,P_n) < \epsilon$. Quindi la funzione è integrabile.

\section{Teorema della media integrale}

\textbf{Enunciato}
Sia $f: [a,b] \to \mathbb{R}$ una funzione continua, allora esiste $c \in [a,b]$ tale che $$f(c)(b-a) = \int_a^b f(x)\dif x$$

\begin{tip}
Esiste un altezza $f(c)$ che moltiplicata per la base $(b-a)$ da il valore dell'area sottesa al grafico della funzione.
\end{tip}

\textbf{Dimostrazione}

La funzione è continua, quindi anche integrabile.

Per il teoream di Weirstrass la funzione ha quindi almeno un punto di massimo $M$ e un punto di minimo $m$.

Di conseguenza possiamo scrivere $$m\cdot (b-a) \leq \int_a^b f(x) \dif x \leq M\cdot (b-a)$$

Dividiamo i membri della disequazione per $b-a$ e otteniamo

$$m \leq \frac{\int_a^b f(x)}{b-a} \dif x \leq M$$

Visto che la funzione è continua essa assume quindi tutti i valori intermedi tra $m$ e $M$, quindi esiste per forza un punto $c$ tale che $f(c)=\frac{\int_a^b f(x)}{b-a}$ cioè $$f(c)\cdot (b-a) = \int_a^b f(x)$$

\section{Funzione integrale}

Data una funzione $f: [a,b] \to \mathbb{R}$, definiamo funzione integrale $F: [a,b] \to \mathbb{R}$ $$F(x) = \int_a^x f(t) \dif t$$

\section{Teorema fondamentale del calcolo integrale}

\subsection{Parte 1}

\textbf{Definizione}

Sia $f$ una funzione integrabile su $[a,b]$ e sia $x_0 \in [a,b]$. Allora la funzione F(x), definita mediante la seguente formula, $$F(x) = \int_{x_0}^x = f(t) \dif t$$ è continua su $[a,b]$

\begin{tip}
Se $f:[a,b] \to \mathbb{R}$ è una funzione integrabile, la funzione integrale $F(x)$ è continua.
\end{tip}

\textbf{Dimostrazione}

\begin{proof}
Essendo integrabile, la funzione $f$ è limitata. Quindi esiste $H \ge 0$ tale che, per ogni $x$,
\begin{equation*}
|f(x)| \le H
\end{equation*}
Consideriamo $F(y)-F(x)$:
\begin{align*}
F(y)-F(x) &= \int_a^y f(t) \, dt - \int_a^x f(t) \, dt \\
&= \int_a^y f(t) \, dt + \int_x^a f(t) \, dt = \int_x^y f(t) \, dt
\end{align*}

Ora:
\begin{equation*}
|F(y)-F(x)| = \left\lvert \int_x^y f(t) \, dt \right\rvert \le \int_x^y |f(t)| \, dt \le \int_x^y H \, dt = (y-x) \cdot H
\end{equation*}
Quanto abbiamo scritto vale ovviamente se $x \le y$. In questo caso
\begin{equation*}
\lim_{y \to x^+} (y-x) \cdot H = 0
\end{equation*}
Sappiamo quindi che
\begin{equation*}
0 \le |F(y) - F(x)| \le (y-x) \cdot H 
\end{equation*}
Quindi, per il teorema del confronto abbiamo che
\begin{equation*}
\lim_{y \to x^+} |F(y)-F(x)| = 0
\end{equation*}

Se fosse invece $x \ge y$, possiamo scambiarli nella penultima equazione e resta
\begin{equation*}
0 \le |F(x) - F(y)| \le (x-y) \cdot H 
\end{equation*}
da cui segue che
\begin{equation*}
\lim_{y \to x^-} (x-y) \cdot H = 0
\end{equation*}
e quindi che
\begin{equation*}
\lim_{y \to x^-} |F(y)-F(x)| = 0
\end{equation*}

In conclusione, unendo i due limiti, trovo che
\begin{equation*}
\lim_{y \to x} |F(y)-F(x)| = 0
\end{equation*}
che è esattamente la definizione di continuità.
\end{proof}

\subsection{Parte 2}

\textbf{Enunciato}

Se $f$ è continua su $[a,b]$ allora $F$ è derivabile su $[a,b]$ e $F'(x) = f(x)$ per ogni $x \in [a,b]$.

\textbf{Dimostrazione}

La funzione integrale esiste, perché $f(x)$ è continua e quindi integrabile.

Sia $x \in [a,b]$ e sia $h$ tale che $x+h \in [a,b]$. Scriviamo il rapporto incrementale in $x$:
\begin{equation*}
\frac{F(x+h)-F(x)}{h}
\end{equation*}
\begin{equation*}
= \frac{1}{h} \left( \int_a^{x+h} f(t) \, dt - \int_a^x f(t) \, dt \right) = \frac{1}{h} \int_x^{x+h} f(t) \, dt
\end{equation*}
Per il teorema della media integrale esiste $y$ compreso tra $x$ e $x+h$ tale che
\begin{equation*}
f(y) \cdot (x+h-x) = \int_x^{x+h} f(t) \, dt
\end{equation*}
Quindi il rapporto incrementale è $f(y) \cdot \frac{h}{h} = f(y)$. Inoltre
\begin{equation*}
\lim_{h \to 0} y = x
\end{equation*}
Quindi
\begin{equation*}
\lim_{h \to 0} \frac{F(x+h)-F(x)}{h} = f(x)
\end{equation*}
Ciò significa che $F(x)$ è derivabile e la sua derivata è $f(x)$.

\subsection{Parte 3}

\textbf{Enunciato}

Sia $f:[a,b] \to \mathbb{R}$ una funzione continua. Sia $G$ una primitiva di $f$. Allora presi due valori $x,y \in [a,b]$
\begin{equation*}
\int_x^y f(t) \, dt = G(y) - G(x)
\end{equation*}

\textbf{Dimostrazione}

Consideriamo una funzione ausiliaria
\begin{equation*}
H(y) = G(y) - \int_x^y f(t) \, dt
\end{equation*}
Sappiamo che $G(y)$, essendo una primitiva, è derivabile e $G'(y) = f(y)$. Inoltre sappiamo che $\int_x^y f(t) \, dt$ è una funzione derivabile di $y$ e la sua derivata è $f(y)$ (per il teorema fondamentale del calcolo II).

Consideriamo la derivata di $H(y)$:
\begin{equation*}
H'(y) = f(y) - f(y) = 0
\end{equation*}
Poiché $H$ ha derivata zero in $[a,b]$, allora è costante in $[a,b]$. In particolare si ha che $H(y) = H(x)$. Si ha che
\begin{equation*}
H(x) = G(x) - \int_x^x f(t) \, dt = G(x)
\end{equation*}
e anche che
\begin{equation*}
G(x) = H(y) = G(y) - \int_x^y f(t) \, dt
\end{equation*}
Quindi
\begin{equation*}
\int_x^y f(t) \, dt = G(y) - G(x)
\end{equation*}